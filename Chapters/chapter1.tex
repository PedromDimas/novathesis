%!TEX root = ../template.tex
%%%%%%%%%%%%%%%%%%%%%%%%%%%%%%%%%%%%%%%%%%%%%%%%%%%%%%%%%%%%%%%%%%%
%% chapter1.tex
%% NOVA thesis document file
%%
%% Chapter with introduction
%%%%%%%%%%%%%%%%%%%%%%%%%%%%%%%%%%%%%%%%%%%%%%%%%%%%%%%%%%%%%%%%%%%

\typeout{NT FILE chapter1.tex}%

\chapter{Introduction}
\label{cha:introduction}




\prependtographicspath{{Chapters/Figures/Covers/}}

\section{Context and Motivation}
\label{sec:context_and_motivation}

In today’s rapidly evolving digital landscape, Identity and Access Management (IAM)  plays a crucial role in securing organizations by managing identity authentication and authorization. However with the proliferation of connected devices, operational technologies (OT) and Internet of Things (IoT) systems, conventional IAM structures are struggling to keep up. Traditionally, Identity and Access Management has relied on centralized systems, which allow administrators to manage users and their roles in the organization. There are, however, a number of weaknesses with these such as poor integration, governance concerns and single points of failure risks.

With organizations increasing their digital footprints and investing in more sophisticated systems, the demand for flexible and safe IAM systems that can adapt to various types of Identify and Access Governance across different industries is increasingly important. The use of centralized IAM systems which has been popular in the past is becoming less popular due to several disadvantages. The volume of IoT devices, the development of cloud environments, and greater emphasis on privacy and data protection is pushing traditional IAM systems towards their limits. The rise of technologies like blockchain Self-Sovereign Identity (SSI) may provide the answers to some of these challenges. The decentralized model of IAM presents the opportunity to provide individuals with greater control over their digital identities, reduce reliance on centralized authorities, and enhance security and privacy in a way that traditional systems cannot. This shift represents a fundamental change in how trust is established in digital interactions and how organizations can manage access to sensitive resources in a rapidly changing technological landscape.

\section{Research Opportunity}
\label{sub:research_opportunity}
While the prospects offered by Decentralized Identity and Access Management (Decentralized IAM) are quite encouraging, the shift from the traditional IAM systems which are centralized to decentralized alternatives is difficult. Some of the key challenges are incorporating existing systems with decentralized technologies, solving governance issues, maintaining consistency of data within the hybrid environment, and adjusting to new legal requirements.

Legacy IAM solutions have been developed without thoughts on integrations and struggle in a completely new world of decentralized identity solutions. In addition, decentralized approaches use emerging technologies like blockchain, which brings additional aspects of scalability, privacy and security to the table, like providing immutable records through ledgers, having publicly available and testable cryptography and strong and proven consensus and synchronization algorithms. Therefore, while looking into the future towards a decentralized IAM landscape, organizations must cope with these complexities while trying to not disturb business operations and adhere to legal requirements in an acceptable fashion.

There is a gap in the literature regarding how organizations can practically move from their legacy IAM systems to decentralized models. Although much has been written about the benefits of decentralized IAM in theory, there’s a solid need for practical models that indicate actionable steps for such a transition. This provides an opening to research how organizations can leap technical, operational, and regulatory hurdles when implementing Decentralized IAM systems.

\section{Goals}
\label{sub:goals}
In this work, we will propose to create and evaluate a unified framework for guiding organizations in the migration from centralized IAM systems toward their decentralized options. It describes the major challenges such an approach has to deal with in detail, mainly: 
\begin{enumerate}
  \item Interoperability: Compatibility of the old systems with the new decentralized technologies.
  \item Governance: New decentralized governance models that ensure easy decision-making and at the same time enforce security and compliance.
  \item Regulatory Compliance: Create strategies to align decentralized IAM strategies with existing laws and regulations.
  \item System Downtime Minimization: Migrate the system smoothly and effectively with a minimum amount of downtime for business operations.
\end{enumerate}