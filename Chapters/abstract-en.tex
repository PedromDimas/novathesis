%!TEX root = ../template.tex
%%%%%%%%%%%%%%%%%%%%%%%%%%%%%%%%%%%%%%%%%%%%%%%%%%%%%%%%%%%%%%%%%%%%
%% abstract-en.tex
%% NOVA thesis document file
%%
%% Abstract in English([^%]*)
%%%%%%%%%%%%%%%%%%%%%%%%%%%%%%%%%%%%%%%%%%%%%%%%%%%%%%%%%%%%%%%%%%%%

\typeout{NT FILE abstract-en.tex}%

In this thesis, we investigate the topic of migration from legacy, centralized Identity and Access Management (IAM) 
systems to Decentralized IAM frameworks. 
Decentralized IAM offers a number of new features, empowering users with control over their identities and 
fostering a more secure and private access management landscape. 
However, integrating these systems with existing IAM infrastructure presents us with multiple challenges.
Beyond minimizing downtime, a successful transition would also require addressing obstacles such as:
\begin{enumerate}
  \item Heterogeneous Systems: Legacy IAM systems often lack interoperability with Decentralized IAM's 
  architecture, requireing careful integration strategies.
  \item Governance Discrepancies: Shifting from a centralized governance model to a decentralized one 
  requires establishing new policies and procedures for user access management, that is compliant with
  existing regulation in the various fields.
  \item User Education and Adoption: Transitioning users from a familiar system to a new one, often 
  requires effective education and support strategies to ensure strong adoption and positive feedback.
\end{enumerate}

This thesis proposes a framework that tackles these challenges, facilitating a seamless migration and integration 
of legacy IAM with Decentralized IAM. This framework will pave the way for a smooth transition that minimizes 
disruption to the user experience, fosters successful integration of the new security model, 
and ensures continued adherence to regulatory compliance requirements.

% Palavras-chave do resumo em Inglês
% \begin{keywords}
% Keyword 1, Keyword 2, Keyword 3, Keyword 4, Keyword 5, Keyword 6, Keyword 7, Keyword 8, Keyword 9
% \end{keywords}
\keywords{
  Identity and Access Management (IAM) \and
  Decentralized IAM \and
  Identity and Access Management framework
}
