%!TEX root = ../template.tex
%%%%%%%%%%%%%%%%%%%%%%%%%%%%%%%%%%%%%%%%%%%%%%%%%%%%%%%%%%%%%%%%%%%%
%% abstract-pt.tex
%% NOVA thesis document file
%%
%% Abstract in Portuguese
%%%%%%%%%%%%%%%%%%%%%%%%%%%%%%%%%%%%%%%%%%%%%%%%%%%%%%%%%%%%%%%%%%%%

\typeout{NT FILE abstract-pt.tex}%

Nesta dissertação, investigamos o tema da migração de sistemas legados e centralizados de Gerenciamento de 
Identidade e Acesso (IAM) para frameworks descentralizados de IAM. 
O IAM descentralizado oferece uma série de novos recursos, capacitando os usuários a ter controle sobre 
suas identidades e promovendo um cenário de gerenciamento de acesso mais seguro e privado.

No entanto, a integração desses sistemas com a infraestrutura IAM existente apresenta vários desafios. 
Além de minimizar o tempo de inatividade, uma transição bem-sucedida também exigirá a abordagem de 
obstáculos como:

\begin{enumerate}
  \item Sistemas Heterogêneos: Sistemas IAM legados geralmente não possuem interoperabilidade com a 
  arquitetura do IAM descentralizado, exigindo estratégias de integração cuidadosas.
  \item Discrepâncias de Governança: A mudança de um modelo de governança centralizado para um 
  descentralizado requer o estabelecimento de novas políticas e procedimentos para o gerenciamento de acesso de usuários, em conformidade com as regulamentações existentes nas diversas áreas.
  \item Educação e Adoção do Usuário: A transição dos usuários de um sistema familiar para um novo, 
  muitas vezes requer estratégias eficazes de educação e suporte para garantir a adoção e feedback positivo.
\end{enumerate}

Esta dissertação propõe uma estrutura que aborda esses desafios, facilitando uma migração e 
integração perfeitas do IAM legado com o IAM descentralizado. 
Essa estrutura irá pavimentar o caminho para uma transição tranquila que minimize a 
interrupção da experiência do usuário, promova a integração bem-sucedida do novo modelo de 
segurança e garanta a adesão contínua aos requisitos de conformidade regulatória.
% E agora vamos fazer um teste com uma quebra de linha no hífen a ver se a \LaTeX\ duplica o hífen na linha seguinte se usarmos \verb+"-+… em vez de \verb+-+.
%
% zzzz zzz zzzz zzz zzzz zzz zzzz zzz zzzz zzz zzzz zzz zzzz zzz zzzz zzz zzzz comentar"-lhe zzz zzzz zzz zzzz
%
% Sim!  Funciona! :)

% Palavras-chave do resumo em Português
% \begin{keywords}
% Palavra-chave 1, Palavra-chave 2, Palavra-chave 3, Palavra-chave 4
% \end{keywords}
\keywords{
  Identity and Access Management (IAM) \and
  Decentralized IAM \and
  Identity and Access Management framework
}
% to add an extra black line
